\documentclass[letterpaper]{article} 
\usepackage[left = 0.5in, right = 0.5in, top = 0.9in, bottom = 0.9in]{geometry}
\usepackage{enumitem}
\usepackage{multicol}
\usepackage[spanish]{babel}
\usepackage[utf8]{inputenc}

\usepackage{amsmath,amssymb,amsthm}
\usepackage{tikz-cd}
\usepackage{mathrsfs}
\usepackage[bbgreekl]{mathbbol}
\usepackage{dsfont}
\usepackage{graphicx}
\graphicspath{{img/}}

\newcommand{\op}{\operatorname}
\newcommand{\Op}{^{\op{op}}}
\newcommand{\scc}{\mathscr C}
\newcommand{\scd}{\mathscr D}
\newcommand{\sce}{\mathscr E}
\newcommand{\sci}{\mathscr I}
\newcommand{\scj}{\mathscr J}
\newcommand{\scx}{\mathscr X}
\newcommand{\var}{\mathrm{Var}}
\newcommand{\Id}{\operatorname{Id}}
\newcommand{\N}{\mathbb N}
\newcommand{\Z}{\mathbb Z}
\newcommand{\Q}{\mathbb{Q}}
\newcommand{\I}{\mathbb{I}}
\newcommand{\R}{\mathbb{R}}
\newcommand{\C}{\mathbb{C}}
\newcommand{\F}{\mathcal{F}}
\newcommand{\G}{\mathcal{G}}
\newcommand{\B}{\mathcal{B}}
\newcommand{\abs}[1]{\left\lvert #1 \right\rvert}
\newcommand{\inv}{^{-1}}
\renewcommand{\to}{\rightarrow}
\newcommand{\ent}{\Longrightarrow}
\newcommand{\E}{\mathbb{E}}
\renewcommand{\P}{\mathbb{P}}
\newcommand{\1}{\mathds{1}}
\renewcommand{\qedsymbol}{$\blacksquare$}

\theoremstyle{definition}
\newtheorem{dfn}{Definición}
\theoremstyle{definition}
\newtheorem{teo}{Teorema}
\theoremstyle{definition}
\newtheorem{cor}{Corolario}
\theoremstyle{definition}
\newtheorem{prop}{Proposición}
\theoremstyle{definition}
\newtheorem{obs}{Observación}


\title{\textbf{}}
\author{Iván Irving Rosas Domínguez}
\date{\today}

\DeclareSymbolFontAlphabet{\mathbbm}{bbold}
\DeclareSymbolFontAlphabet{\mathbb}{AMSb}
\DeclareMathSymbol\bbDelta  \mathord{bbold}{"01}

\begin{document}
\maketitle

%\begin{abstract}
%\end{abstract}

\begin{enumerate}
    \item[\textbf{1.}] Definir la cdf inversa generalizada $F_X^{-}$ y demostrar que en 
    el caso de variables aleatorias continuas esta coincide con la inversa usual. demostrar
    además que en general para simular de $X$ podemos simular $u\sim U(0,1)$ y $F_X^{-}(u)$ se 
    distribuye como $X$ [1 punto]
    \item[\textbf{2.}]Implementar el siguiente algoritmo para simular variables aleatorias 
    uniformes:
    \[
    X_i=107374182x_{i-1}+104420x_{i-5} \qquad (mod 2^{31}-1),
    \]
    regresa $x_i$ y recorrer el estado, esto es, $x_{j-1}=x_{j}$; $j=1,2,3,4,5;$ ¿parecen $U(0,1)$?[1 punto]
    \item[\textbf{3.}] ¿Cuál es el algoritmo que usa $scipy.stats.uniform$ para generar números aleatorios?
    ¿Cómo se pone la semilla? ¿Y en R?
    \item[\textbf{4.}] ¿En $scipy$ qué funciones hay para simular una variable aleatoria
    genérica discreta? ¿tienen preproceso? [1 punto]
    \item[\textbf{5.}] Implementar el algoritmo Adaptive Rejection Sampling y simular una $Gamma(2,1)$, 10,000 muestras. ¿Cuando
    es conveniente dejar de adaptar la envolvente? (ver alg. A.7, p. 54 Robert y Casella, 2da ed.) [6 puntos]
\end{enumerate}
\end{document}