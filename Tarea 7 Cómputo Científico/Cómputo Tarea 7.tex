\documentclass[letterpaper]{article} 
\usepackage[left = 0.5in, right = 0.5in, top = 0.9in, bottom = 0.9in]{geometry}
\usepackage{enumitem}
\usepackage{multicol}
\usepackage[spanish]{babel}
\usepackage[utf8]{inputenc}

\usepackage{amsmath,amssymb,amsthm}
\usepackage{tikz-cd}
\usepackage{mathrsfs}
\usepackage[bbgreekl]{mathbbol}
\usepackage{dsfont}
\usepackage{graphicx}
\graphicspath{{img/}}

\newcommand{\op}{\operatorname}
\newcommand{\Op}{^{\op{op}}}
\newcommand{\scc}{\mathscr C}
\newcommand{\scd}{\mathscr D}
\newcommand{\sce}{\mathscr E}
\newcommand{\sci}{\mathscr I}
\newcommand{\scj}{\mathscr J}
\newcommand{\scx}{\mathscr X}
\newcommand{\var}{\mathrm{Var}}
\newcommand{\Id}{\operatorname{Id}}
\newcommand{\N}{\mathbb N}
\newcommand{\Z}{\mathbb Z}
\newcommand{\Q}{\mathbb{Q}}
\newcommand{\I}{\mathbb{I}}
\newcommand{\R}{\mathbb{R}}
\newcommand{\C}{\mathbb{C}}
\newcommand{\F}{\mathcal{F}}
\newcommand{\G}{\mathcal{G}}
\newcommand{\B}{\mathcal{B}}
\newcommand{\abs}[1]{\left\lvert #1 \right\rvert}
\newcommand{\inv}{^{-1}}
\renewcommand{\to}{\rightarrow}
\newcommand{\ent}{\Longrightarrow}
\newcommand{\E}{\mathbb{E}}
\renewcommand{\P}{\mathbb{P}}
\newcommand{\1}{\mathds{1}}
\renewcommand{\qedsymbol}{$\blacksquare$}

\theoremstyle{definition}
\newtheorem{dfn}{Definición}
\theoremstyle{definition}
\newtheorem{teo}{Teorema}
\theoremstyle{definition}
\newtheorem{cor}{Corolario}
\theoremstyle{definition}
\newtheorem{prop}{Proposición}
\theoremstyle{definition}
\newtheorem{obs}{Observación}


\title{\textbf{Cómputo Científico\\
Tarea 7\\
MCMC: Metrópolis-Hastings II}}
\author{Iván Irving Rosas Domínguez}
\date{\today}

\DeclareSymbolFontAlphabet{\mathbbm}{bbold}
\DeclareSymbolFontAlphabet{\mathbb}{AMSb}
\DeclareMathSymbol\bbDelta  \mathord{bbold}{"01}

\begin{document}
\maketitle

% \begin{abstract}
% \end{abstract}

\textbf{Con el algoritmo Metrópolis-Hastings (MH), simular lo siguiente:}
\begin{itemize}
    \item[\textbf{1.}] Sean $x_i\sim\Gamma(\alpha,\beta); \ i\in \{1,...,n\}$. Simular datos $x_i$ 
    con $\alpha=3$ y $\beta=100$ considerando los casos $n=4$ y $n=30$.

    Con $\alpha\sim U(1,4)$, $\beta\sim \exp(1)$ distribuciones a priori, se tiene la
    posterior 
    \[
    f(\alpha,\beta|\bar{x})\propto\frac{\beta^{n\alpha}}{\Gamma(\alpha)}r_1^{\alpha-1}e^{-\beta(r_2+1)}, \qquad 1\leq \alpha\leq 4, \quad \beta>1.    
    \]
    con $r_2=\displaystyle \sum_{i=1}^nx_i$ y $r_1=\displaystyle\prod_{i=1}^{n}x_i$.

    En ambos casos, grafica los contornos para visualizar dónde está concentrada la posterior.

    Utilizar la propuesta 
    \[
    q \left( \begin{pmatrix}
        \alpha_p\\
        \beta_p
    \end{pmatrix} \Bigg|
    \begin{pmatrix}
        \alpha\\
        \beta
    \end{pmatrix}
    \right)= \begin{pmatrix}
        \alpha\\
        \beta
    \end{pmatrix}
    +
    \begin{pmatrix}
        \varepsilon_1\\
        \varepsilon_2
    \end{pmatrix},    
    \]
    donde 
    \[
    \begin{pmatrix}
        \varepsilon_1\\
        \varepsilon_2
    \end{pmatrix}    
    \sim 
    \mathcal{N}_2 \left( \begin{pmatrix}
        0\\
        0
    \end{pmatrix},
    \begin{pmatrix}
        \sigma_1^2 & 0\\
        0 & \sigma_2^2
    \end{pmatrix}\right).
    \]
    \item[\textbf{2.}] Simular de la distribución $\Gamma(\alpha,1)$ con la propuesta $\Gamma(\left[\alpha\right],1)$,
    donde $\left[\alpha\right]$ denota la parte entera de $\alpha$.
    \newline

    Además, realizar el siguiente experimento: poner como punto inicial $x_0=900$ y graficar 
    la evolución de la cadena, es decir, $f(X_t)$ vs $t$.
    \item[\textbf{3.}] Implementar Random Walk Metrópolis Hastings (RWMH) donde la distribución
    objetivo es $\mathcal{N}_2(\mu,\Sigma)$, con 
    \[
    \mu=\begin{pmatrix}
        3\\
        5
    \end{pmatrix} \ \text{ y } \ 
    \begin{pmatrix}
        1 & 0.9\\
        0.9 & 1
    \end{pmatrix}.    
    \]
    Utilizar como propuesta $\varepsilon_t\sim \mathcal{N}_2(\textbf{0},\sigma I)$. ¿Cómo elegir 
    $\sigma$ para que la cadena sea eficiente? ¿Qué consecuencias tiene la elección de $\sigma$?

    Como experimento, elige como punto inicial $x_0= \begin{pmatrix}
        1000\\
        1
    \end{pmatrix}$ y comenta los resultados.\\

    \textbf{Para todos los incisos del ejercicio anterior:}
    \begin{itemize}
        \item Establece cual es tu distribución inicial.
        \item Grafica la evolución de la cadena.
        \item Indica cuál es el Burn-in.
        \item Comenta qué tan eficiente es la cadena.
        \item Implementa el algoritmo MH considerando una propuesta diferente.
    \end{itemize}
\end{itemize}


\end{document}  